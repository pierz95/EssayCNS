
%% bare_conf_compsoc.tex
%% V1.4b
%% 2015/08/26
%% by Michael Shell
%% See:
%% http://www.michaelshell.org/
%% for current contact information.
%%
%% This is a skeleton file demonstrating the use of IEEEtran.cls
%% (requires IEEEtran.cls version 1.8b or later) with an IEEE Computer
%% Society conference paper.
%%
%% Support sites:
%% http://www.michaelshell.org/tex/ieeetran/
%% http://www.ctan.org/pkg/ieeetran
%% and
%% http://www.ieee.org/

%%*************************************************************************
%% Legal Notice:
%% This code is offered as-is without any warranty either expressed or
%% implied; without even the implied warranty of MERCHANTABILITY or
%% FITNESS FOR A PARTICULAR PURPOSE! 
%% User assumes all risk.
%% In no event shall the IEEE or any contributor to this code be liable for
%% any damages or losses, including, but not limited to, incidental,
%% consequential, or any other damages, resulting from the use or misuse
%% of any information contained here.
%%
%% All comments are the opinions of their respective authors and are not
%% necessarily endorsed by the IEEE.
%%
%% This work is distributed under the LaTeX Project Public License (LPPL)
%% ( http://www.latex-project.org/ ) version 1.3, and may be freely used,
%% distributed and modified. A copy of the LPPL, version 1.3, is included
%% in the base LaTeX documentation of all distributions of LaTeX released
%% 2003/12/01 or later.
%% Retain all contribution notices and credits.
%% ** Modified files should be clearly indicated as such, including  **
%% ** renaming them and changing author support contact information. **
%%*************************************************************************


% *** Authors should verify (and, if needed, correct) their LaTeX system  ***
% *** with the testflow diagnostic prior to trusting their LaTeX platform ***
% *** with production work. The IEEE's font choices and paper sizes can   ***
% *** trigger bugs that do not appear when using other class files.       ***                          ***
% The testflow support page is at:
% http://www.michaelshell.org/tex/testflow/



\documentclass[conference,compsoc]{IEEEtran}
% Some/most Computer Society conferences require the compsoc mode option,
% but others may want the standard conference format.
%
% If IEEEtran.cls has not been installed into the LaTeX system files,
% manually specify the path to it like:
% \documentclass[conference,compsoc]{../sty/IEEEtran}





% Some very useful LaTeX packages include:
% (uncomment the ones you want to load)


% *** MISC UTILITY PACKAGES ***
%
%\usepackage{ifpdf}
% Heiko Oberdiek's ifpdf.sty is very useful if you need conditional
% compilation based on whether the output is pdf or dvi.
% usage:
% \ifpdf
%   % pdf code
% \else
%   % dvi code
% \fi
% The latest version of ifpdf.sty can be obtained from:
% http://www.ctan.org/pkg/ifpdf
% Also, note that IEEEtran.cls V1.7 and later provides a builtin
% \ifCLASSINFOpdf conditional that works the same way.
% When switching from latex to pdflatex and vice-versa, the compiler may
% have to be run twice to clear warning/error messages.






% *** CITATION PACKAGES ***
%
\ifCLASSOPTIONcompsoc
  % IEEE Computer Society needs nocompress option
  % requires cite.sty v4.0 or later (November 2003)
  \usepackage[nocompress]{cite}
\else
  % normal IEEE
  \usepackage{cite}
\fi
% cite.sty was written by Donald Arseneau
% V1.6 and later of IEEEtran pre-defines the format of the cite.sty package
% \cite{} output to follow that of the IEEE. Loading the cite package will
% result in citation numbers being automatically sorted and properly
% "compressed/ranged". e.g., [1], [9], [2], [7], [5], [6] without using
% cite.sty will become [1], [2], [5]--[7], [9] using cite.sty. cite.sty's
% \cite will automatically add leading space, if needed. Use cite.sty's
% noadjust option (cite.sty V3.8 and later) if you want to turn this off
% such as if a citation ever needs to be enclosed in parenthesis.
% cite.sty is already installed on most LaTeX systems. Be sure and use
% version 5.0 (2009-03-20) and later if using hyperref.sty.
% The latest version can be obtained at:
% http://www.ctan.org/pkg/cite
% The documentation is contained in the cite.sty file itself.
%
% Note that some packages require special options to format as the Computer
% Society requires. In particular, Computer Society  papers do not use
% compressed citation ranges as is done in typical IEEE papers
% (e.g., [1]-[4]). Instead, they list every citation separately in order
% (e.g., [1], [2], [3], [4]). To get the latter we need to load the cite
% package with the nocompress option which is supported by cite.sty v4.0
% and later.





% *** GRAPHICS RELATED PACKAGES ***
%
\ifCLASSINFOpdf
   \usepackage[pdftex]{graphicx}
  % declare the path(s) where your graphic files are
  \graphicspath{{../images/}}
  % and their extensions so you won't have to specify these with
  % every instance of \includegraphics
   \DeclareGraphicsExtensions{.pdf,.jpeg,.png}
\else
  % or other class option (dvipsone, dvipdf, if not using dvips). graphicx
  % will default to the driver specified in the system graphics.cfg if no
  % driver is specified.
  % \usepackage[dvips]{graphicx}
  % declare the path(s) where your graphic files are
  % \graphicspath{{../eps/}}
  % and their extensions so you won't have to specify these with
  % every instance of \includegraphics
  % \DeclareGraphicsExtensions{.eps}
\fi
% graphicx was written by David Carlisle and Sebastian Rahtz. It is
% required if you want graphics, photos, etc. graphicx.sty is already
% installed on most LaTeX systems. The latest version and documentation
% can be obtained at: 
% http://www.ctan.org/pkg/graphicx
% Another good source of documentation is "Using Imported Graphics in
% LaTeX2e" by Keith Reckdahl which can be found at:
% http://www.ctan.org/pkg/epslatex
%
% latex, and pdflatex in dvi mode, support graphics in encapsulated
% postscript (.eps) format. pdflatex in pdf mode supports graphics
% in .pdf, .jpeg, .png and .mps (metapost) formats. Users should ensure
% that all non-photo figures use a vector format (.eps, .pdf, .mps) and
% not a bitmapped formats (.jpeg, .png). The IEEE frowns on bitmapped formats
% which can result in "jaggedy"/blurry rendering of lines and letters as
% well as large increases in file sizes.
%
% You can find documentation about the pdfTeX application at:
% http://www.tug.org/applications/pdftex





% *** MATH PACKAGES ***
%
%\usepackage{amsmath}
% A popular package from the American Mathematical Society that provides
% many useful and powerful commands for dealing with mathematics.
%
% Note that the amsmath package sets \interdisplaylinepenalty to 10000
% thus preventing page breaks from occurring within multiline equations. Use:
%\interdisplaylinepenalty=2500
% after loading amsmath to restore such page breaks as IEEEtran.cls normally
% does. amsmath.sty is already installed on most LaTeX systems. The latest
% version and documentation can be obtained at:
% http://www.ctan.org/pkg/amsmath





% *** SPECIALIZED LIST PACKAGES ***
%
%\usepackage{algorithmic}
% algorithmic.sty was written by Peter Williams and Rogerio Brito.
% This package provides an algorithmic environment fo describing algorithms.
% You can use the algorithmic environment in-text or within a figure
% environment to provide for a floating algorithm. Do NOT use the algorithm
% floating environment provided by algorithm.sty (by the same authors) or
% algorithm2e.sty (by Christophe Fiorio) as the IEEE does not use dedicated
% algorithm float types and packages that provide these will not provide
% correct IEEE style captions. The latest version and documentation of
% algorithmic.sty can be obtained at:
% http://www.ctan.org/pkg/algorithms
% Also of interest may be the (relatively newer and more customizable)
% algorithmicx.sty package by Szasz Janos:
% http://www.ctan.org/pkg/algorithmicx




% *** ALIGNMENT PACKAGES ***
%
%\usepackage{array}
% Frank Mittelbach's and David Carlisle's array.sty patches and improves
% the standard LaTeX2e array and tabular environments to provide better
% appearance and additional user controls. As the default LaTeX2e table
% generation code is lacking to the point of almost being broken with
% respect to the quality of the end results, all users are strongly
% advised to use an enhanced (at the very least that provided by array.sty)
% set of table tools. array.sty is already installed on most systems. The
% latest version and documentation can be obtained at:
% http://www.ctan.org/pkg/array


% IEEEtran contains the IEEEeqnarray family of commands that can be used to
% generate multiline equations as well as matrices, tables, etc., of high
% quality.




% *** SUBFIGURE PACKAGES ***
%\ifCLASSOPTIONcompsoc
%  \usepackage[caption=false,font=footnotesize,labelfont=sf,textfont=sf]{subfig}
%\else
%  \usepackage[caption=false,font=footnotesize]{subfig}
%\fi
% subfig.sty, written by Steven Douglas Cochran, is the modern replacement
% for subfigure.sty, the latter of which is no longer maintained and is
% incompatible with some LaTeX packages including fixltx2e. However,
% subfig.sty requires and automatically loads Axel Sommerfeldt's caption.sty
% which will override IEEEtran.cls' handling of captions and this will result
% in non-IEEE style figure/table captions. To prevent this problem, be sure
% and invoke subfig.sty's "caption=false" package option (available since
% subfig.sty version 1.3, 2005/06/28) as this is will preserve IEEEtran.cls
% handling of captions.
% Note that the Computer Society format requires a sans serif font rather
% than the serif font used in traditional IEEE formatting and thus the need
% to invoke different subfig.sty package options depending on whether
% compsoc mode has been enabled.
%
% The latest version and documentation of subfig.sty can be obtained at:
% http://www.ctan.org/pkg/subfig




% *** FLOAT PACKAGES ***
%
%\usepackage{fixltx2e}
% fixltx2e, the successor to the earlier fix2col.sty, was written by
% Frank Mittelbach and David Carlisle. This package corrects a few problems
% in the LaTeX2e kernel, the most notable of which is that in current
% LaTeX2e releases, the ordering of single and double column floats is not
% guaranteed to be preserved. Thus, an unpatched LaTeX2e can allow a
% single column figure to be placed prior to an earlier double column
% figure.
% Be aware that LaTeX2e kernels dated 2015 and later have fixltx2e.sty's
% corrections already built into the system in which case a warning will
% be issued if an attempt is made to load fixltx2e.sty as it is no longer
% needed.
% The latest version and documentation can be found at:
% http://www.ctan.org/pkg/fixltx2e


%\usepackage{stfloats}
% stfloats.sty was written by Sigitas Tolusis. This package gives LaTeX2e
% the ability to do double column floats at the bottom of the page as well
% as the top. (e.g., "\begin{figure*}[!b]" is not normally possible in
% LaTeX2e). It also provides a command:
%\fnbelowfloat
% to enable the placement of footnotes below bottom floats (the standard
% LaTeX2e kernel puts them above bottom floats). This is an invasive package
% which rewrites many portions of the LaTeX2e float routines. It may not work
% with other packages that modify the LaTeX2e float routines. The latest
% version and documentation can be obtained at:
% http://www.ctan.org/pkg/stfloats
% Do not use the stfloats baselinefloat ability as the IEEE does not allow
% \baselineskip to stretch. Authors submitting work to the IEEE should note
% that the IEEE rarely uses double column equations and that authors should try
% to avoid such use. Do not be tempted to use the cuted.sty or midfloat.sty
% packages (also by Sigitas Tolusis) as the IEEE does not format its papers in
% such ways.
% Do not attempt to use stfloats with fixltx2e as they are incompatible.
% Instead, use Morten Hogholm'a dblfloatfix which combines the features
% of both fixltx2e and stfloats:
%
% \usepackage{dblfloatfix}
% The latest version can be found at:
% http://www.ctan.org/pkg/dblfloatfix




% *** PDF, URL AND HYPERLINK PACKAGES ***
%
%\usepackage{url}
% url.sty was written by Donald Arseneau. It provides better support for
% handling and breaking URLs. url.sty is already installed on most LaTeX
% systems. The latest version and documentation can be obtained at:
% http://www.ctan.org/pkg/url
% Basically, \url{my_url_here}.




% *** Do not adjust lengths that control margins, column widths, etc. ***
% *** Do not use packages that alter fonts (such as pslatex).         ***
% There should be no need to do such things with IEEEtran.cls V1.6 and later.
% (Unless specifically asked to do so by the journal or conference you plan
% to submit to, of course. )


% correct bad hyphenation here
\hyphenation{op-tical net-works semi-conduc-tor}


\begin{document}
%
% paper title
% Titles are generally capitalized except for words such as a, an, and, as,
% at, but, by, for, in, nor, of, on, or, the, to and up, which are usually
% not capitalized unless they are the first or last word of the title.
% Linebreaks \\ can be used within to get better formatting as desired.
% Do not put math or special symbols in the title.
\title{Secure static content delivery \\for content distribution network using blockchain technology}


% author names and affiliations
% use a multiple column layout for up to three different
% affiliations
\author{\IEEEauthorblockN{Pier Paolo Tricomi}
\IEEEauthorblockA{Student ID: 1179740}
}


% conference papers do not typically use \thanks and this command
% is locked out in conference mode. If really needed, such as for
% the acknowledgment of grants, issue a \IEEEoverridecommandlockouts
% after \documentclass

% for over three affiliations, or if they all won't fit within the width
% of the page (and note that there is less available width in this regard for
% compsoc conferences compared to traditional conferences), use this
% alternative format:
% 
%\author{\IEEEauthorblockN{Michael Shell\IEEEauthorrefmark{1},
%Homer Simpson\IEEEauthorrefmark{2},
%James Kirk\IEEEauthorrefmark{3}, 
%Montgomery Scott\IEEEauthorrefmark{3} and
%Eldon Tyrell\IEEEauthorrefmark{4}}
%\IEEEauthorblockA{\IEEEauthorrefmark{1}School of Electrical and Computer Engineering\\
%Georgia Institute of Technology,
%Atlanta, Georgia 30332--0250\\ Email: see http://www.michaelshell.org/contact.html}
%\IEEEauthorblockA{\IEEEauthorrefmark{2}Twentieth Century Fox, Springfield, USA\\
%Email: homer@thesimpsons.com}
%\IEEEauthorblockA{\IEEEauthorrefmark{3}Starfleet Academy, San Francisco, California 96678-2391\\
%Telephone: (800) 555--1212, Fax: (888) 555--1212}
%\IEEEauthorblockA{\IEEEauthorrefmark{4}Tyrell Inc., 123 Replicant Street, Los Angeles, California 90210--4321}}




% use for special paper notices
%\IEEEspecialpapernotice{(Invited Paper)}




% make the title area
\maketitle

% As a general rule, do not put math, special symbols or citations
% in the abstract
\begin{abstract}
A Content Distribution Network (CDN) is a new kind of network with the goal to distribute services and contents spatially relative to end-users to provide high availability and high performance. Several replicas of the Origin Server are used to reach this goal, but trust issues are now involved both between servers and among client and server. In this work is presented a new method to provide secure static content delivery using blockchain, a growing technology with the capability to ensure reliability and trust without a central authority.\\
Moreover, a prototype of the sistem has been developed on Ethereum private network, in order to test the feasibility. The test shows the goodness of the system, and the possibility to create a new content distribution model on the Internet.  
\end{abstract}

% no keywords




% For peer review papers, you can put extra information on the cover
% page as needed:
% \ifCLASSOPTIONpeerreview
% \begin{center} \bfseries EDICS Category: 3-BBND \end{center}
% \fi
%
% For peerreview papers, this IEEEtran command inserts a page break and
% creates the second title. It will be ignored for other modes.
\IEEEpeerreviewmaketitle



\section{Introduction}
% no \IEEEPARstart
Contents distribution is one of the most important aspects of the Internet. To improve the distribution the Content Distribution Networks (CDNs) are born. CDNs provide high availability and high performance because many replicas (Edge Servers) of the Origin Server are spatially distributed to faster fulfill a request. Many services like CoralCDN\cite{freedman2004democratizing} CoDeeN\cite{wang2002effectiveness} provide the possibility to create a CDN starting from the Origin Server, replicating contents and distributing them to end users from the best Edge Server, likely the geographically nearest. A typical scenario can be seen in Figure \ref{fig:CDN}, the Origin Server has more replicas to distribute contents.

\begin{figure}[!t]
\centering
\includegraphics[width=2.2in]{images/cdn.png}

\caption{Typical CDN scenario where Origin Server has many Edge Servers to fulfill the clients requests.}
\label{fig:CDN}
\end{figure} 

In this scenario, two new trust issues are involved. First of all, an attacker could modify the contents from the Origin Server to the Edge Server, thus if the indirectly attacked Edge Server serves that modified content it will be labeled as a misbehaving replica. Secondly, if an Edge Server is not directly managed by the owner of the Origin Server, it can serve different content like stale content or outright modified content, adding ads for example.
Furthermore in this architecture the Origin Server is a single point of failure. If the Origin server is compromised all the Servers will misbehave.\\
The main contribute of this work is two-fold.\\First, it is suggested a new architecture to overcome the trust issue between servers, mainteining the CDN properties. Second, the proposed system provides a secure method to deliver static contents, ensuring the integrity with small effort on clients and Network. 

\section{Related Works}
Checking the integrity of data retrieved from untrusted servers is a crucial problem nowadays. The typical approach for preventing contents tampering between clients and servers is to encrypt the end-to-end connection, for example, using the SSL protocol.
This negates the functionality of the CDN, because the connection always requires the Origin Server, and the network is no more distributed. For static content, Merkle tree authentication \cite{bayardo2005merkle} is a possible solution, where the server signs the content which is verified by the client. Also digital rights management schemes allows clients to verify data from untrusted server \cite{adelsbach2005towards}. For dynamic content, \cite{chi2002xml}\cite{orman2001data} propose the use of XML-based rules for managing the content, but it has to be limited to be easily verified by
a client.\\
In peer-to-peer CDNs the problem is even more significant. LOCKSS\cite{maniatis2003preserving} uses voting system for content integrity. 
Repeate the execution to detect misbehavior has been used in Rx\cite{qin2005rx} and in Vigilante \cite{costa2005vigilante} to discover bugs and worms respectively. These approachs require significant overhead on the client. In Pioneer\cite{seshadri2005pioneer} the verify effort is on the dispatcher, a trusted platform, but since the proof of correctness is extremely time sensitive it is not suitable for large scale systems.
Finally Repeate and Compare system \cite{michalakis2007ensuring} is for both static and dynamic contents, it requires the repetition of the content to another replica and compares the results to detect misbehaving replicas. However, the process has significant overhead, and the verify requests may overwhelm the network, thus only a fraction of the contents are verified.\\
In this work blockchain technology is used. The structure of blockchain ensure trust between untrusted nodes without central authority. A system of contents distribution built over blockchain inherits all its benefits, making simple to share contents and verify them between clients and servers. 


\section{Blockchain Technology}
After the rise of Bitcoin \cite{nakamoto2008bitcoin}, its underlying structure, the blockchain technology, has been applied to a variety of usecases ranging from authentication \cite{sundararajanonline} (using Ethereum and Smart Contracts) to medical reports\cite{azaria2016medrec}. Blockchain is based on append-only ledger, a growing list of records (called blocks) which are linked and secured using cryptography. The ledger can be viewed by all participating nodes and the updates are permitted only after the consensum of the network. Each block typically contains a cryptographic hash of the previous block, a timestamp and transaction data. Thanks to its structure and the proof-of-work required to create a new block, a blockchain is inherently resistant to modification of the data. A simple example of blockchain is shown in Figure \ref{fig:blockchain}.\\
\begin{figure}[!h]
	\centering
	\includegraphics[width=3.8in]{images/blockchain.jpg}
	
	\caption{Example of blockchain. Transactions are stored in the blocks of the Ledger. Blocks are linked by hashes.}
	\label{fig:blockchain}
\end{figure} 

The presented system relies on Ethereum, an open-source, public, blockchain-based distributed computing platform, which allows to execute smart contracts, immutable programs always visible from the community.


\subsection{Ethereum}
Ethereum \cite{wood2014ethereum} is described as ”a transactional singleton machine with shared-state”. 
The distributed ledger can be view as a distributed virtual machine which records transactions and state changes.
The blockchain is kept alive by the miners, nodes which have three main tasks:
\begin{itemize}
	\item Verifing the transactions being sent, through the consensum algorithm which is similar to other blockchains, removing the need of central authority;
	\item Checking if the sender has sufficient gas, the Ethereum's native cryptocurrency, to transfer to the receiver;
	\item Running the function called by the sender and thus to modify the blockchain state accordingly. 
\end{itemize}

In regards to the third point, we see the most important feature of Ethereum: encoding smart contracts. 


\subsubsection{Smart Contracts}

Smart contracts describe functions, which can be called by nodes participating in the blockchain, and states which can be changed by the nodes. 
The functions run in the Ethereum Virtual Machine (EVM)
which is described as a ”\textit{quasi}-Turing complete” computer able to execute code of any complexity. The \textit{quasi} qualification comes from the
fact that the computation is intrinsically bounded to \textit{gas}, which limits the total amount of computation
done.
If the gas transferred to the miner for the function execution is less than the required amount, then the transaction is not mined.  
\\The contract can use memory and storage to save data. It can use any amount of memory (paying gas) during executing its code, but when execution stops, the entire content of the memory is wiped. The storage on the other hand is persisted into the blockchain itself. It can be changed, but all the changes will be recorded in the ledger.\\
In the presented sytem, thanks of its persistence and (im)mutability, storage will be used to store and share contents. 


\subsection{Benefits}
In this paragraph, the benefits of using blockchain are discussed. Building the content delivery system on the blockchain it will inherit all the benefits.
\begin{itemize}
	\item \textbf{Decentralization:} Consensus mechanism is used to agree on the validity of transactions, thus there is no need for a trusted third party;
	
	\item \textbf{Transparency and trust:} Blockchains are shared and visible from all the partecipants. This makes the system transparent and as a result trust is established;
	
	\item \textbf{Immutability:} Once the data has been written into the blockchain, it is extremely difficult to change it back. In Ethereum if the state is changed all the updates are stored in the ledger. If only few user are allowed to change the state, the state remains immutable as long as they don't change it;
	
	\item \textbf{High availability:} As the system is based on several nodes in a peer-to-peer network, and the data is replicated and updated on each node, the network as a whole continues to work even if nodes leave or become unavailable;
	
	\item \textbf{Highly secure:} All transactions on a blockchain are cryptographically secured and provide integrity.
	\vspace{10pt}
\end{itemize} 

\section{System Design}
In this section is presented the developed system to distribute and check integrity of the contents. Firstly an high-level overview is shown. Secondly more details about the architecture are provided. Lastly benefits and drowbacks of the system are discussed.

\subsection{Overview}
Figure \ref{fig:SystemOverview} shows high-level view of how the presented system distributes the contents on the Servers and permits to Client to check integrity.


\begin{figure}[!h]
	\centering
	\includegraphics[width=3.8in]{images/SystemOverview.png}
	
	\caption{Overview of the content delivery and check system.}
	\label{fig:SystemOverview}
\end{figure} 

The system wants to resolve trust issues between Origin and Edge Servers, since the last can receive altered contents, and between Servers and Clients, as Server can send different content from the requested one. The key idea is to use two different blockchains: the \textit{Contents Blockchain} and the \textit{Hashes Blockchain}.
\subsubsection{Contents Blockchain}
The proposal is to build the CDN on a blockchain, the \textit{Contents Blockchain}. This means each Edge Server is a node of the blockchain. The Origin Server is no more required, since it is possible to add a content from every node (Edge Server) as long as it is authorizated, and the content will be automatically distributed on each Server thanks to blockchain technology. To authorize the upload of the contents an user system is used. Since the blockchain is always visible from every partecipant, and every partecipant has a public address, which is a public and private key pair, it is enough to store in the blockchain the addresses of who can upload the contents, and after checking who is requesting the upload, the transaction can be executed or not. Two kind of user are permitted: Admin and Superadmin. Admins can upload/remove/edit the contents while Superadmin can also create other Admins. 

\subsubsection{Hashes Blockchain}
The \textit{Hashes Blockchain} is the key to check the integrity of received contents. Thanks to hash functions is really simple to check message integrity \cite{tsudik1992message}. An hash function is any function that can be used to map data of arbitrary size to data of fixed size. A cryptographic hash function allows one to easily verify that some input data maps to a given hash value, but if the input data is unknown, it is extremely difficult to reconstruct it by knowing the stored hash value. These property are very useful for the goal. Every time a content is uploaded, its hash is calculated by the Server and stored in the \textit{Hashes Blockchain}, where the partecipants are both Servers and Clients. After the client receives the requested content it can easily check its integrity calculating its hash and comparing it with the one stored in the blockchain. If the hashes are equals the content is right, otherwise something wrong happend. Since storing the hash of a content requires the consensum of the network, it is always correct, and there is no possibility for a server to change it without noticing the others nodes. 
	
\subsection{Typical Usage}
A typical usecase is shown in Figure \ref{fig:SystemOverview}. First an Admin wants to upload a new content. After the system checks and authorizes him, a new upload request is sent among the \textit{Contents Blockchain}. When the network approves the request, a copy of the content is distributed on each server (according to blockchain technology) and its hash is calculated and stored in the \textit{Hashes Blockchain}. At any time a client can ask for that content. When the content is received the client calculate its hash with the same hash function used by the Server and compares the obtained hash with the calculated one. If they are equals the integrity is verified, otherwise the client can reject the content and ask to another server.

\subsection{Benefits and Drawbacks}

KEY IDEAS: 
Simplification of current paradigms
The current model in many industries such as finance or health is rather disorganized, wherein multiple entities maintain their own databases and data sharing can become very difficult due to the disparate nature of the systems. But as a blockchain can serve as a single shared ledger among interested parties, this can result in simplifying this model by reducing the complexity of managing the separate systems maintained by each entity.


In the financial industry, especially in post-trade settlement functions, blockchain can play a vital role by allowing the quicker settlement of trades as it does not require a lengthy process of verification, reconciliation, and clearance because a single version of agreed upon data is already available on a shared ledger between financial organizations.


\section{Implementation}
introduzione su solidity e caratteristiche del pc, geth rpc testnet con più nodi, easy miners etc

\subsection{Contents Blockchain}
\subsection{Hashes Blockchain}

\section{Future Works}

% An example of a floating figure using the graphicx package.
% Note that \label must occur AFTER (or within) \caption.
% For figures, \caption should occur after the \includegraphics.
% Note that IEEEtran v1.7 and later has special internal code that
% is designed to preserve the operation of \label within \caption
% even when the captionsoff option is in effect. However, because
% of issues like this, it may be the safest practice to put all your
% \label just after \caption rather than within \caption{}.
%
% Reminder: the "draftcls" or "draftclsnofoot", not "draft", class
% option should be used if it is desired that the figures are to be
% displayed while in draft mode.
%
%\begin{figure}[!t]
%\centering
%\includegraphics[width=2.5in]{myfigure}
% where an .eps filename suffix will be assumed under latex, 
% and a .pdf suffix will be assumed for pdflatex; or what has been declared
% via \DeclareGraphicsExtensions.
%\caption{Simulation results for the network.}
%\label{fig_sim}
%\end{figure}

% Note that the IEEE typically puts floats only at the top, even when this
% results in a large percentage of a column being occupied by floats.


% An example of a double column floating figure using two subfigures.
% (The subfig.sty package must be loaded for this to work.)
% The subfigure \label commands are set within each subfloat command,
% and the \label for the overall figure must come after \caption.
% \hfil is used as a separator to get equal spacing.
% Watch out that the combined width of all the subfigures on a 
% line do not exceed the text width or a line break will occur.
%
%\begin{figure*}[!t]
%\centering
%\subfloat[Case I]{\includegraphics[width=2.5in]{box}%
%\label{fig_first_case}}
%\hfil
%\subfloat[Case II]{\includegraphics[width=2.5in]{box}%
%\label{fig_second_case}}
%\caption{Simulation results for the network.}
%\label{fig_sim}
%\end{figure*}
%
% Note that often IEEE papers with subfigures do not employ subfigure
% captions (using the optional argument to \subfloat[]), but instead will
% reference/describe all of them (a), (b), etc., within the main caption.
% Be aware that for subfig.sty to generate the (a), (b), etc., subfigure
% labels, the optional argument to \subfloat must be present. If a
% subcaption is not desired, just leave its contents blank,
% e.g., \subfloat[].


% An example of a floating table. Note that, for IEEE style tables, the
% \caption command should come BEFORE the table and, given that table
% captions serve much like titles, are usually capitalized except for words
% such as a, an, and, as, at, but, by, for, in, nor, of, on, or, the, to
% and up, which are usually not capitalized unless they are the first or
% last word of the caption. Table text will default to \footnotesize as
% the IEEE normally uses this smaller font for tables.
% The \label must come after \caption as always.
%
%\begin{table}[!t]
%% increase table row spacing, adjust to taste
%\renewcommand{\arraystretch}{1.3}
% if using array.sty, it might be a good idea to tweak the value of
% \extrarowheight as needed to properly center the text within the cells
%\caption{An Example of a Table}
%\label{table_example}
%\centering
%% Some packages, such as MDW tools, offer better commands for making tables
%% than the plain LaTeX2e tabular which is used here.
%\begin{tabular}{|c||c|}
%\hline
%One & Two\\
%\hline
%Three & Four\\
%\hline
%\end{tabular}
%\end{table}


% Note that the IEEE does not put floats in the very first column
% - or typically anywhere on the first page for that matter. Also,
% in-text middle ("here") positioning is typically not used, but it
% is allowed and encouraged for Computer Society conferences (but
% not Computer Society journals). Most IEEE journals/conferences use
% top floats exclusively. 
% Note that, LaTeX2e, unlike IEEE journals/conferences, places
% footnotes above bottom floats. This can be corrected via the
% \fnbelowfloat command of the stfloats package.




\section{Conclusion}
The conclusion goes here.




% conference papers do not normally have an appendix



% use section* for acknowledgment
\ifCLASSOPTIONcompsoc
  % The Computer Society usually uses the plural form
  \section*{Acknowledgments}
\else
  % regular IEEE prefers the singular form
  \section*{Acknowledgment}
\fi


The authors would like to thank...





% trigger a \newpage just before the given reference
% number - used to balance the columns on the last page
% adjust value as needed - may need to be readjusted if
% the document is modified later
%\IEEEtriggeratref{8}
% The "triggered" command can be changed if desired:
%\IEEEtriggercmd{\enlargethispage{-5in}}

% references section

% can use a bibliography generated by BibTeX as a .bbl file
% BibTeX documentation can be easily obtained at:
% http://mirror.ctan.org/biblio/bibtex/contrib/doc/
% The IEEEtran BibTeX style support page is at:
% http://www.michaelshell.org/tex/ieeetran/bibtex/
\bibliographystyle{IEEEtran}
% argument is your BibTeX string definitions and bibliography database(s)
\bibliography{references}
%
% <OR> manually copy in the resultant .bbl file
% set second argument of \begin to the number of references
% (used to reserve space for the reference number labels box)

%\begin{thebibliography}{1}

%\bibitem{IEEEhowto:kopka}
%H.~Kopka and P.~W. Daly, \emph{A Guide to \LaTeX}, 3rd~ed.\hskip 1em plus
%  0.5em minus 0.4em\relax Harlow, England: Addison-Wesley, 1999.

%\end{thebibliography}




% that's all folks
\end{document}


